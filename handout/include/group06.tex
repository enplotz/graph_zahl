%!TEX root = ../main.tex
\section{InfiniteGraph}

\subsection{General}

\paragraph{Product Website} \url{http://www.inifinitegraph.com}

\paragraph{Project Group}
Jens \textit{Erat}
Todor \textit{Georgiev}
J\"{u}rgen \textit{H\"{o}lsch}
Manuel \textit{Hotz}
Florian \textit{Junghaus}
Jens \textit{Metzner}
Marc Steffen \textit{Spicker}

\begin{itemize}
\item \emph{Product website}
\item \emph{Short general product description (not a text block from Wikipedia or the product website)}
\item \emph{Authors of this profile}
\end{itemize}

\subsection{Data Model}
\begin{itemize}
\item \emph{Precise description of the data model, especially in terms of differences from the \dquote{standard} models presented in the lecture.}
\item \emph{Detailed summary of the basic data manipulation API, i.e., features to create, retrieve, update and delete data items.}
\end{itemize}

\subsection{Query Support}
\begin{itemize}
\item \emph{What types of queries are supported, i.e., point, range, navigation, and/or arbitrary?}
\item \emph{What is the query language of the system? Is it declarative, functional, algebraic and/or imperative?}
\item \emph{Are queries automatically optimized?}
\end{itemize}

\subsection{Indexes}
\begin{itemize}
\item \emph{What index structures are available?}
\item \emph{What can be indexed? What can be a key? What can be a value?}
\item \emph{How are indexes managed, i.e., manually or automatically?}
\end{itemize}

\subsection{Storage}
\begin{itemize}
\item \emph{How is data stored physically, i.e., file format and organization, data catalog, etc.?}
\item \emph{Where can data be stored physically, i.e., disk or file storage, in-memory (RAM), flash or SSD, traditional database, cloud storage (GFS, HDFS, S3), etc.?}
\item \emph{How is data accessed, i.e., how do read and write operations work?}
\end{itemize}

\subsection{Transactions and Concurrency Control}
\begin{itemize}
\item \emph{How would you classify the system according to CAP and PAC/ELC?}
\item \emph{Does the system support transactions?}
\item \emph{How are transactions implemented, i.e., locks, OCC, MVCC, etc.?}
\item \emph{What consistency guarantees are given?}
\end{itemize}

\subsection{Scalability, Availability and Fault Tolerance}
\begin{itemize}
\item \emph{What types of scalability are supported, i.e., horizontal and/or vertical?}
\item \emph{What types of data partitioning are supported, i.e., horizontal (sharding) and/or vertical?}
\item \emph{What types of data replication are supported, i.e., synchronous and/or asynchronous?}
\item \emph{How is scalability achieved?}
\item \emph{How is fault tolerance achieved?}
\item \emph{How is availability achieved?}
\end{itemize}

\subsection{Platform/Deployment}
\begin{itemize}
\item \emph{What cloud infrastructures are supported?}
\item \emph{What deployment scenarios are supported, i.e., embedded, client/server, multi-core CPU, cloud, etc.?}
\item \emph{Language bindings?}
\item \emph{Communication protocols, i.e., JSON, REST, etc.?}
\end{itemize}

\subsection{Applications}
\begin{itemize}
\item \emph{Have you found any major applications that use this data store and, if so, which ones?}
\item \emph{Is there a typical kind of applications that use this data store and, if so, how would you characterize them in terms of requirements?}
\end{itemize}

\subsection{Miscellaneous}
\begin{itemize}
\item \emph{Are there any features or interesting aspects that you could not list under the previous points and, if so, which one?}
\end{itemize}

\newpage