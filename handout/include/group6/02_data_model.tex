%!TEX root = ../../main.tex

\emph{Precise description of the data model, especially in terms of differences from the \dquote{standard} models presented in the lecture.}

The data model corresponds to a \textbf{property graph} (similar to the lecture)
\begin{itemize}
	\item Nodes and Edges are Objects that can contain arbitrary properties.
	\item Edges can be incoming, outgoing or bidirectional and always contain a weight
	\item Java base classes for nodes (BaseNode) and edges (BaseEdge)
	\begin{itemize}
		\item[$\rightarrow$] define identity
		\item[$\rightarrow$] can be extended, attributes added
		\item[$\rightarrow$] properties can be indexed (by \dquote{own} index
		or by alternative indices like \textit{Apache Lucene}.
	\end{itemize}
\end{itemize}

\emph{Detailed summary of the basic data manipulation API, i.e., features to create, retrieve, update and delete data items.}

The basic data manipulation API:

\begin{itemize}

	\item Database creation / opening	
	\begin{lstlisting}[caption=(Creating / opening a DB)]
	GraphFactory.create(dbName);
	GraphDatabase db = GraphFactory.open(dbName);
	\end{lstlisting}

	\item Database closing / deletion
	\begin{lstlisting}[caption=(Closing / deleting a DB)]
	db.close();
	GraphFactory.delete(dbName);
	\end{lstlisting}

	\item Vertex / Edge creation
	\begin{lstlisting}[caption=(Vertex / Edge creation)]
	class Person extends BaseVertex;
	class Friends extends BaseEdge;

	long v1 = db.addVertex(person1);
	long v2 = db.addVertex(person2);
	long e1 = person1.addEdge(friends1, person2, 
	EdgeKind.BIDIRECTIONAL, weight);
	//long i = db.addEdge(BaseEdge, BaseVertex, 
	//BaseVertex, EdgeKind, weight);
	\end{lstlisting}

	\item Vertex / Edge retrieval
	\begin{lstlisting}[caption=(Vertex / Edge retrieval)]
	db.getVertex(ID);
	db.getVertices();
	db.getEdge(ID);
	db.getEdges();
	\end{lstlisting}

	\item Vertex / Edge removal
	\begin{lstlisting}[caption=(Vertex / Edge removal)]
	db.removeVertex(person1);
	db.removeVertexByID(ID);
	db.removeEdge(friends1);
	db.removeEdgeByID(ID);
	\end{lstlisting}

	\item Updates
	Updates are up to the user!

	\begin{lstlisting}[caption=(User defined update method)]
	person1.setName("Michael");
	\end{lstlisting}

	\textbf{BUT:} Adding / removing additional properties not that easy 	$\rightarrow$ new subclass needed!
	\begin{lstlisting}[caption=(Subclass of Person with additional "age" property)]	
	class PersonWithAge extends Person;

	void setAge(int age);
	\end{lstlisting}

\item Handles:
	\begin{itemize}
		\item fast access to an object without retrieving the complete object
		\item stores graph related information like neighbors
		\item can be customized  by implementing the \textit{VertexHandle}
		(edge count, edges), \textit{EdgeHandle} (weight, kind, peer)
	\end{itemize}
\end{itemize}