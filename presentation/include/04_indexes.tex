%!TEX root = ../main.tex

\begin{frame}
\frametitle{What can be indexed?}
\begin{itemize}
		\item Attributes of edges and vertices can be indexed
		\item All standard data types (e.g. int, long, float, double, string) can be used as a key		
\end{itemize}
\end{frame} 

\begin{frame}
\frametitle{Types of indexes}
There are three types of indexes:

\begin{enumerate}
	\item \textbf{Graph Index}
	\item \textbf{Generic Index}
	\item \textbf{Lucene Index}
\end{enumerate}
\end{frame} 

\begin{frame}
\frametitle{Graph Index}
\begin{itemize}
	\item Supports regular expression queries \pause
	\item Objects of a given type are automatically added to the index \pause
	\item The user only has to define the class and the fields that should be indexed \pause
	\item Can index multiple attributes \pause
	\item Is updated automatically
\end{itemize}
\end{frame} 

\begin{frame}
\frametitle{Generic Index}
\begin{itemize}
	\item Provides range-search lookups on a single attribute \pause
	\item No support for regular expression queries \pause
	\item The user has to add objects manually to the index \pause
	\item Can only have a single attribute as a key \pause
	\item Has to be updated manually by the application 
\end{itemize}
\end{frame} 

\begin{frame}
\frametitle{Lucene Index}
\begin{itemize}
	\item Uses Apache Lucene\footnote{\url{http://lucene.apache.org/core/}} full-text indexing and search capabilities
	\item Has to be updated manually by the application
\end{itemize}
\end{frame}