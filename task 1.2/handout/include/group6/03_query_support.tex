%!TEX root = ../../main.tex

%####################
\begin{itemize}
\item Single selection by the name of the vertex possible
\begin{lstlisting}[caption=(Point query)]
Person michael = db.getNamedVertex("Michael");
\end{lstlisting}
\item Predicate Query Language (PQL\footnote{http://wiki.infinitegraph.com/3.1/w/docs/PQL/jgdObjectQualification.html}) is used to scan Database (regular expression also allowed)
\begin{lstlisting}[caption=(Range query)]
Query<Person> personQuery = db.createQuery(
	Person.class.getName(), 
	"name=='Michael' && age<100");
\end{lstlisting}
\end{itemize}
(Queries on the field "name" of the \textit{Person} class will have the optimal performance if there exist a graph index in this field.) \\

%####################

A Navigation API also exist and supports:
\begin{itemize}
\item simple breadth first navigation
\item simple depth first navigation
\item own path and result qualifiers are allowed
\begin{itemize}
\item \textit{path qualifier} is used validate a path
\item \textit{result qualifier} is used to limit the results to a set that follows given rules
\end{itemize}
\item can fire own Callback function that processes the result
\end{itemize}
\begin{lstlisting}[caption=(Creating a simple Navigator)]
Navigator navi = michael.navigate(
Guide.SIMPLE_DEPTH_FIRST, Qualifier.FOREVER, 
Qualifier.ANY, new NavigationResultHandler());
\end{lstlisting}

%#####################

A simple result qualifier 
\begin{lstlisting}[caption=(Result qualifier)]
class IsMichaelQualifier implements Qualifier
{
    String name = new String("Michael");
    public boolean qualify(Path currentPath)
    {
        return name.equals(((Person) (currentPath
            .getFinalHop().getVertex())).getName());
    }
}
\end{lstlisting}
%####################