%!TEX root = ../../main.tex

\begin{itemize}
\item \emph{What index structures are available?}
	
	There are three types of indexes:
	\begin{enumerate}
		\item \textbf{Graph Index}:
			\begin{itemize}
				\item Supports regular expression queries
				\item Objects of a given type are automatically added to the index
				\item The user only has to define the class and the fields that should be indexed
			\end{itemize}
		\item \textbf{Generic Index}:
			\begin{itemize}
				\item Provides range-search lookups on a single attribute
				\item No support for regular expression queries
				\item The user has to add objects manually to the index
			\end{itemize}
		\item \textbf{Lucene Index}:
			\begin{itemize}
				\item Uses Apache Lucene$^{\tiny{TM}}$ full-text indexing and search capabilities
			\end{itemize}
	\end{enumerate}
\item \emph{What can be indexed? What can be a key? What can be a value?}
	\begin{itemize}
		\item Attributes of edges and vertices can be indexed
		\item A Graph Index can index multiple attributes
		\item A Generic Index can only have a single attribute as a key
		\item In a Lucene Index multiple attributes can be defined, but in a query only one attribute can be used 
		\item All standard types (e.g. int, long, float, double, string) can be used as a key
		\item Edges and vertices can be values of an index
	\end{itemize}
\item \emph{How are indexes managed, i.e., manually or automatically?}
	\begin{itemize}
		\item A Graph Index is updated automatically
		\item Generic or Lucene Indexes have to be updated manually by the application 
	\end{itemize}
\end{itemize}