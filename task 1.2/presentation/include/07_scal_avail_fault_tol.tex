%!TEX root = ../main.tex
\begin{frame}
\frametitle{Scalability}
\begin{itemize}
  \item very Good scalability
         \item The goal is to create a graph database with virtually unlimited scalability
         \item It is said that infinite graph can search billions of connections and relationships in massive volumes of complex data
\end{itemize}
\end{frame} 


\begin{frame}
\frametitle{What types of scalability are supported, i.e., horizontal and/or vertical?}
\begin{itemize}
	\item Supports horizontal scalability
\end{itemize}
\end{frame} 

\begin{frame}
\frametitle{What types of data partitioning are supported, i.e., horizontal (sharding) and/or vertical?}
\begin{itemize}
	\item Autonomous partition structure	
	\item Autonomous partitions are independent from network and system failures
	\item Each has own lock server and a complete architecture
	\item Each contains references to other partitions
\end{itemize}
\end{frame} 

\begin{frame}
\frametitle{What types of data replication are supported, i.e., synchronous and/or asynchronous?}
\begin{itemize}
	\item Synchronous data replication
	\item If one database image is updated, a min number of other replicas are updated also
	\item min number of images represents consistent view of he database
\end{itemize}
\end{frame} 

\begin{frame}
\frametitle{How is scalability achieved?}
\begin{itemize}
	\item Distributed data tier supports parallel IO
	\item Distributed cash/processor tier
         \item Quorum protocol
\end{itemize}
\end{frame} 

\begin{frame}
\frametitle{How is fault tolerance achieved?}
\begin{itemize}
\item Objectivity/DB Fault Tolerant Option
\item Services are distributed and relocated so that each partition is self-sufficient in case of failure in other partitions
	
\end{itemize}
\end{frame} 
\begin{frame}

\frametitle{How is availability achieved?}
\begin{itemize}
\item Strive for high availability 
\item Reads requested information from closest copy
\item Multiple network or machine failures possible before databases become unavailable
\end{itemize}
\end{frame} 
